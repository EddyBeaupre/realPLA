\documentclass[a4paper,oneside]{memoir}
%\documentclass[ebook]{memoir}

\setlength{\trimtop}{0pt}
\setlength{\trimedge}{\stockwidth}
\addtolength{\trimedge}{-\paperwidth}
\settypeblocksize{24cm}{14cm}{*}
\setulmargins{3cm}{*}{*}
\setlrmargins{4cm}{*}{*}
\setmarginnotes{17pt}{51pt}{\onelineskip}
\setheadfoot{\onelineskip}{2\onelineskip}
\setheaderspaces{*}{2\onelineskip}{*}
\checkandfixthelayout

\clubpenalty = 10000
\widowpenalty = 10000
\displaywidowpenalty = 10000

\usepackage[utf8x]{inputenx}
%\usepackage{selinput}
\usepackage{microtype}
\usepackage{graphicx}
\usepackage{hyperref} % muss letztes package sein

\hypersetup{pdfborder=0 0 0}


\title{The C64 PLA}
\author{
Thomas 'skoe' Giesel
(skoe@directbox.com)
}

\begin{document}
\emergencystretch=0.15\hsize

%%%%%%%%%%%%%%%%%%%%%%%%%%%%%%%%%%%%%%%%%%%%%%%%%%%%%%%%%%%%%%%%%%%%%%%%
\pagestyle{empty}
\begin{center}

\vspace*{5cm}
\textsc{\huge Designing a Real KERNAL Cartridge}\\[2cm]
{\large Thomas 'skoe' Giesel}

\vfill

% Bottom of the page
{\large First Edition \\[0.5cm] \today}

\end{center}
%%%%%%%%%%%%%%%%%%%%%%%%%%%%%%%%%%%%%%%%%%%%%%%%%%%%%%%%%%%%%%%%%%%%%%%%

\clearpage
\tableofcontents

\chapter{Introduction}

\section{About this Document}

The PLA (programmable logic array) in the Commodore 64 is used to create
chip select signals from various other signals, e.g. from the current
address. These signals control which chip is to be connected to the data
bus. Therefore the PLA is responsible to define the memory map of the C64.

Understanding the logical function of the PLA is important for programmers,
for hardware and emulator developers. To implement a drop-in replacement for
original PLA chips it is also important to have a close look at the
electrical properties of the original PLAs. This document addresses both
topics.

In the last chapter an implementation of the PLA in hardware is introduced
which can be built with parts being in production at the time of
writing. This implementation aims at highest compatibility possible while
still using low-cost parts only.

Why is it neccesary to write a document about the PLA in 2012? Things
already documented are spread over many articles, buried in various forums
and mailing lists and printed in old dusty books. This document aims to
collect existing knowledge and to add new measurement and research results.
Please look out for new revisions of this document. If you have new or
additional information relevant for this document, please contact me.

\section{Acknowledgements}

I wish to thank the engineers of Commodore for the interesting memory maps
achieved by using the PLA. Sadly, you forgot an external KERNAL ;)

Furthermore I'dl like to express respect to all guys who examined the PLA
before and published their results, namely William Levak, Jens Schönfeld,
Marko Mäkelä, Andreas Boose and Mark Smith.

All PLA logic contained in this document and used for the implementation is
based on the JEDEC file by Raymond Jett of arcadecomponents.com. It was
created by reading a C64 PLA with a TopMAX programmer and Max Loader
software. The file is available at \cite{AC12} and in the source code
repository of the realPLA implementation (link: todo).

Segher Boessenkool described in a mail to the cbm-hackers mailing list how
the JEDEC file can be read directly. Thanks to him I was able to implemented
a small tool which converted the JEDEC file to VHDL equations directly.

\chapter{The Logic in the PLA}

\section{History}

The memory maps in \cite {PRG83} show all possible combinations as they are
seen by the CPU, including Ultimax mode. However, some details cannot be
taken from \cite{PRG83}, especially memory configurations for VIC-II
accesses, identified by the signals AEC and BA. Furthermore, some information
shown there is just plain wrong, which is the case for the column
labeled XXXX.

Probably the earliest full reverse engineering of the logic programmed to
the PLA was written by William Levak in 1986. It was published in \cite
{Lev86}. Unfortunately it has a mistake in the memory maps, although the
logic table is correct: The memory holes in Ultimax mode (\#GAME = 0, \#EXROM
= 1) are incorrectly labeled with "RAM".

Some years later, in 1994, Jens Schönfeld captured a full truth
table from a PLA. He, Marko Mäkelä, Andreas Boose and others \cite
{PLA95} investigated that truth table and derived equations from it.
This was the first time (to the knowledge of the author) the logic
from the PLA was documented openly without errors.

Mark Smith took a programmer and read out an 82S100 in 1995. The
results confirmed the work from one year ago. Later different PLAs
with different part numbers by different manufacturers found in C64s
have been read out. No differences have been found.

\chapter{Hardware}
\section{Types of PLAs found in C64s}

The well known PLA type 82S100 by Signetics/Phillips can be found in
many early C64s. There are also similar devices with different names
made by other manufacturers. They can be recognized by their high
idle current, they draw about 100 mA when all inputs are high, and
the outputs are without load.

Later chips labelled with MOS 906114-01 were used, which was a mask
programmable NMOS version directly derived from the 82S100
structure. After some manufacturing problems at the beginning
(1983/1984?) they were used until the PLA in the C64 was replaced by
a higher integrated IC.

\section{Electrical Properties}

In this section the results of measurements of some electrical properties of
original PLAs and different PLA replacements are shown. Their meaning for
the inner workings of the C64 is discussed.

\subsection{Devices under Test}

\subsection{Power Consumption}

One of the most impressive properties of the PLA is: It gets quite hot for
this little work it has to do. To get an impression about their static power
dissipation, I mounted the devices on a breadboard. \#OE was grounded, so
the outputs were enabled. All inputs were pulled high, the outputs didn't
have a load. The current at VCC was measured with a multimeter. Additionally
the temperature was measured with an infrared thermometer in the middle of
the chip package's surface after a few minutes. Table XXX shows the results.


\chapter{A PLA replacement: realPLA}


\appendix
\chapter{C64 Memory Configurations}

The Commodore 64 memory configurations are contained in various books and
other documents. Many of these tables are incomplete, some of them are
flawed. Certainly there are also several complete and correct ones out there.
I do not claim that there's anything new here.

For the sake of completeness of an C64 PLA article they are listed here once
more. The format of the tables is inspired by \cite{Lev86}, but they have
been written in a slightly more compact way. I did my best to double und triple
check them against the equations to make sure they are correct and complete.

When the VIC-II pulls down BA to take over the bus e.g. for a so-called
badline, there are three cycles left for the CPU. In these cycles up to
three write accesses can take place, which is the maximum number of
consecutive write accesses a 6510 does. If such a write access addresses the
I/O area, the PLA selects the I/O chips. However, the CPU stops its work
before the first read access is started during these three cycles. Therefore
there are up to three dummy read accesses on the bus until the VIC-II takes
over control. To make sure that these dummy reads will never accidentally
acknowledge an interrupt, they are redirected to RAM by the PLA.

\begin{table}[!h]
    \centering
    \begin{tabularx}{0.9\textwidth}{>{\centering}X|c|c|c|c|c}
        \toprule
        \multicolumn{6}{c}{\#LORAM 1, \#HIRAM 1, \#GAME 1, \#EXROM 1} \\
        \midrule
            & \multicolumn{2}{c|}{\#CHAREN 1} & \multicolumn{2}{c|}{\#CHAREN 0} & \\
        Address         & CPU R & CPU W & CPU R & CPU W & VIC R \\
        \midrule
        \texttt{\$F000 - \$FFFF} & KERNAL   & ram   & KERNAL   & ram       & ram   \\
        \texttt{\$E000 - \$EFFF} & KERNAL   & ram   & KERNAL   & ram       & ram   \\
        \texttt{\$D000 - \$DFFF} & I/O\footnotemark[1] & I/O & CHAR     & ram       & ram   \\
        \texttt{\$C000 - \$CFFF} & ram      & ram   & ram      & ram       & ram   \\
        \texttt{\$B000 - \$BFFF} & BASIC    & ram   & BASIC    & ram       & ram   \\
        \texttt{\$A000 - \$AFFF} & BASIC    & ram   & BASIC    & ram       & ram   \\
        \texttt{\$9000 - \$9FFF} & ram      & ram   & ram      & ram       & CHAR  \\
        \texttt{\$8000 - \$8FFF} & ram      & ram   & ram      & ram       & ram   \\
        \texttt{\$7000 - \$7FFF} & ram      & ram   & ram      & ram       & ram   \\
        \texttt{\$6000 - \$6FFF} & ram      & ram   & ram      & ram       & ram   \\
        \texttt{\$5000 - \$5FFF} & ram      & ram   & ram      & ram       & ram   \\
        \texttt{\$4000 - \$4FFF} & ram      & ram   & ram      & ram       & ram   \\
        \texttt{\$3000 - \$3FFF} & ram      & ram   & ram      & ram       & ram   \\
        \texttt{\$2000 - \$2FFF} & ram      & ram   & ram      & ram       & ram   \\
        \texttt{\$1000 - \$1FFF} & ram      & ram   & ram      & ram       & CHAR  \\
        \texttt{\$0000 - \$0FFF} & ram      & ram   & ram      & ram       & ram   \\
        \bottomrule
    \end{tabularx}
    \caption{C64 memory configurations with LHGX = 1111}
    \label{tab:mem1111}
\end{table}

\footnotetext[1]{RAM for CPU idle read when BA is low}

\begin{table}[!h]
    \centering
    \begin{tabularx}{0.9\textwidth}{>{\centering}X|c|c|c|c|c}
        \toprule
        \multicolumn{6}{c}{\#LORAM 0, \#HIRAM 1, \#GAME 1, \#EXROM x} \\
        \midrule
            & \multicolumn{2}{c|}{\#CHAREN 1} & \multicolumn{2}{c|}{\#CHAREN 0} & \\
        Address         & CPU R & CPU W & CPU R & CPU W & VIC R \\
        \midrule
        \texttt{\$F000 - \$FFFF} & KERNAL   & ram   & KERNAL   & ram       & ram   \\
        \texttt{\$E000 - \$EFFF} & KERNAL   & ram   & KERNAL   & ram       & ram   \\
        \texttt{\$D000 - \$DFFF} & I/O\footnotemark[1] & I/O & CHAR     & ram       & ram   \\
        \texttt{\$C000 - \$CFFF} & ram      & ram   & ram      & ram       & ram   \\
        \texttt{\$B000 - \$BFFF} & ram      & ram   & ram      & ram       & ram   \\
        \texttt{\$A000 - \$AFFF} & ram      & ram   & ram      & ram       & ram   \\
        \texttt{\$9000 - \$9FFF} & ram      & ram   & ram      & ram       & CHAR  \\
        \texttt{\$8000 - \$8FFF} & ram      & ram   & ram      & ram       & ram   \\
        \texttt{\$7000 - \$7FFF} & ram      & ram   & ram      & ram       & ram   \\
        \texttt{\$6000 - \$6FFF} & ram      & ram   & ram      & ram       & ram   \\
        \texttt{\$5000 - \$5FFF} & ram      & ram   & ram      & ram       & ram   \\
        \texttt{\$4000 - \$4FFF} & ram      & ram   & ram      & ram       & ram   \\
        \texttt{\$3000 - \$3FFF} & ram      & ram   & ram      & ram       & ram   \\
        \texttt{\$2000 - \$2FFF} & ram      & ram   & ram      & ram       & ram   \\
        \texttt{\$1000 - \$1FFF} & ram      & ram   & ram      & ram       & CHAR  \\
        \texttt{\$0000 - \$0FFF} & ram      & ram   & ram      & ram       & ram   \\
        \bottomrule
    \end{tabularx}
    \caption{C64 memory configurations with LHGX = 011x}
    \label{tab:mem011x}
\end{table}

\begin{table}[!h]
    \centering
    \begin{tabularx}{0.9\textwidth}{>{\centering}X|c|c|c|c|c}
        \toprule
        \multicolumn{6}{c}{\#LORAM 1, \#HIRAM 0, \#GAME 1, \#EXROM x} \\
        \multicolumn{6}{c}{\#LORAM 1, \#HIRAM 0, \#GAME 0, \#EXROM 0} \\
        \midrule
            & \multicolumn{2}{c|}{\#CHAREN 1} & \multicolumn{2}{c|}{\#CHAREN 0} & \\
        Address         & CPU R & CPU W & CPU R & CPU W & VIC R \\
        \midrule
        \texttt{\$F000 - \$FFFF} & ram      & ram   & ram      & ram       & ram   \\
        \texttt{\$E000 - \$EFFF} & ram      & ram   & ram      & ram       & ram   \\
        \texttt{\$D000 - \$DFFF} & I/O\footnotemark[1] & I/O & CHAR     & ram       & ram   \\
        \texttt{\$C000 - \$CFFF} & ram      & ram   & ram      & ram       & ram   \\
        \texttt{\$B000 - \$BFFF} & ram      & ram   & ram      & ram       & ram   \\
        \texttt{\$A000 - \$AFFF} & ram      & ram   & ram      & ram       & ram   \\
        \texttt{\$9000 - \$9FFF} & ram      & ram   & ram      & ram       & CHAR  \\
        \texttt{\$8000 - \$8FFF} & ram      & ram   & ram      & ram       & ram   \\
        \texttt{\$7000 - \$7FFF} & ram      & ram   & ram      & ram       & ram   \\
        \texttt{\$6000 - \$6FFF} & ram      & ram   & ram      & ram       & ram   \\
        \texttt{\$5000 - \$5FFF} & ram      & ram   & ram      & ram       & ram   \\
        \texttt{\$4000 - \$4FFF} & ram      & ram   & ram      & ram       & ram   \\
        \texttt{\$3000 - \$3FFF} & ram      & ram   & ram      & ram       & ram   \\
        \texttt{\$2000 - \$2FFF} & ram      & ram   & ram      & ram       & ram   \\
        \texttt{\$1000 - \$1FFF} & ram      & ram   & ram      & ram       & CHAR  \\
        \texttt{\$0000 - \$0FFF} & ram      & ram   & ram      & ram       & ram   \\
        \bottomrule
    \end{tabularx}
    \caption{C64 memory configurations with LHGX = 101x or 1000}
    \label{tab:mem101x}
\end{table}

\begin{table}[!h]
    \centering
    \begin{tabularx}{0.65\textwidth}{>{\centering}X|c|c|c}
        \toprule
        \multicolumn{4}{c}{\#LORAM 0, \#HIRAM 0, \#GAME 1, \#EXROM x} \\
        \multicolumn{4}{c}{\#LORAM 0, \#HIRAM 0, \#GAME 0, \#EXROM 0} \\
        \midrule
            & \multicolumn{2}{c|}{\#CHAREN x} & \\
        Address         & CPU R & CPU W & VIC R \\
        \midrule
        \texttt{\$F000 - \$FFFF} & ram      & ram       & ram   \\
        \texttt{\$E000 - \$EFFF} & ram      & ram       & ram   \\
        \texttt{\$D000 - \$DFFF} & ram      & ram       & ram   \\
        \texttt{\$C000 - \$CFFF} & ram      & ram       & ram   \\
        \texttt{\$B000 - \$BFFF} & ram      & ram       & ram   \\
        \texttt{\$A000 - \$AFFF} & ram      & ram       & ram   \\
        \texttt{\$9000 - \$9FFF} & ram      & ram       & CHAR  \\
        \texttt{\$8000 - \$8FFF} & ram      & ram       & ram   \\
        \texttt{\$7000 - \$7FFF} & ram      & ram       & ram   \\
        \texttt{\$6000 - \$6FFF} & ram      & ram       & ram   \\
        \texttt{\$5000 - \$5FFF} & ram      & ram       & ram   \\
        \texttt{\$4000 - \$4FFF} & ram      & ram       & ram   \\
        \texttt{\$3000 - \$3FFF} & ram      & ram       & ram   \\
        \texttt{\$2000 - \$2FFF} & ram      & ram       & ram   \\
        \texttt{\$1000 - \$1FFF} & ram      & ram       & CHAR  \\
        \texttt{\$0000 - \$0FFF} & ram      & ram       & ram   \\
        \bottomrule
    \end{tabularx}
    \caption{C64 memory configurations with LHGX = 001x or 0000}
    \label{tab:mem001x}
\end{table}

\begin{table}[!h]
    \centering
    \begin{tabularx}{0.9\textwidth}{>{\centering}X|c|c|c|c|c}
        \toprule
        \multicolumn{6}{c}{\#LORAM 1, \#HIRAM 1, \#GAME 0, \#EXROM 0} \\
        \midrule
            & \multicolumn{2}{c|}{\#CHAREN 1} & \multicolumn{2}{c|}{\#CHAREN 0} & \\
        Address         & CPU R & CPU W & CPU R & CPU W & VIC R \\
        \midrule
        \texttt{\$F000 - \$FFFF} & KERNAL   & ram   & KERNAL   & ram       & ram   \\
        \texttt{\$E000 - \$EFFF} & KERNAL   & ram   & KERNAL   & ram       & ram   \\
        \texttt{\$D000 - \$DFFF} & I/O\footnotemark[1] & I/O & CHAR     & ram       & ram   \\
        \texttt{\$C000 - \$CFFF} & ram      & ram   & ram      & ram       & ram   \\
        \texttt{\$B000 - \$BFFF} & ROMH     & ram   & ROMH     & ram       & ram   \\
        \texttt{\$A000 - \$AFFF} & ROMH     & ram   & ROMH     & ram       & ram   \\
        \texttt{\$9000 - \$9FFF} & ROML     & ram   & ROML     & ram       & CHAR  \\
        \texttt{\$8000 - \$8FFF} & ROML     & ram   & ROML     & ram       & ram   \\
        \texttt{\$7000 - \$7FFF} & ram      & ram   & ram      & ram       & ram   \\
        \texttt{\$6000 - \$6FFF} & ram      & ram   & ram      & ram       & ram   \\
        \texttt{\$5000 - \$5FFF} & ram      & ram   & ram      & ram       & ram   \\
        \texttt{\$4000 - \$4FFF} & ram      & ram   & ram      & ram       & ram   \\
        \texttt{\$3000 - \$3FFF} & ram      & ram   & ram      & ram       & ram   \\
        \texttt{\$2000 - \$2FFF} & ram      & ram   & ram      & ram       & ram   \\
        \texttt{\$1000 - \$1FFF} & ram      & ram   & ram      & ram       & CHAR  \\
        \texttt{\$0000 - \$0FFF} & ram      & ram   & ram      & ram       & ram   \\
        \bottomrule
    \end{tabularx}
    \caption{C64 memory configurations with LHGX = 1100}
    \label{tab:mem1100}
\end{table}

\begin{table}[!h]
    \centering
    \begin{tabularx}{0.9\textwidth}{>{\centering}X|c|c|c|c|c}
        \toprule
        \multicolumn{6}{c}{\#LORAM 0, \#HIRAM 1, \#GAME 0, \#EXROM 0} \\
        \midrule
            & \multicolumn{2}{c|}{\#CHAREN 1} & \multicolumn{2}{c|}{\#CHAREN 0} & \\
        Address         & CPU R & CPU W & CPU R & CPU W & VIC R \\
        \midrule
        \texttt{\$F000 - \$FFFF} & KERNAL   & ram   & KERNAL   & ram       & ram   \\
        \texttt{\$E000 - \$EFFF} & KERNAL   & ram   & KERNAL   & ram       & ram   \\
        \texttt{\$D000 - \$DFFF} & I/O\footnotemark[1] & I/O & CHAR     & ram       & ram   \\
        \texttt{\$C000 - \$CFFF} & ram      & ram   & ram      & ram       & ram   \\
        \texttt{\$B000 - \$BFFF} & ROMH     & ram   & ROMH     & ram       & ram   \\
        \texttt{\$A000 - \$AFFF} & ROMH     & ram   & ROMH     & ram       & ram   \\
        \texttt{\$9000 - \$9FFF} & ram      & ram   & ram      & ram       & CHAR  \\
        \texttt{\$8000 - \$8FFF} & ram      & ram   & ram      & ram       & ram   \\
        \texttt{\$7000 - \$7FFF} & ram      & ram   & ram      & ram       & ram   \\
        \texttt{\$6000 - \$6FFF} & ram      & ram   & ram      & ram       & ram   \\
        \texttt{\$5000 - \$5FFF} & ram      & ram   & ram      & ram       & ram   \\
        \texttt{\$4000 - \$4FFF} & ram      & ram   & ram      & ram       & ram   \\
        \texttt{\$3000 - \$3FFF} & ram      & ram   & ram      & ram       & ram   \\
        \texttt{\$2000 - \$2FFF} & ram      & ram   & ram      & ram       & ram   \\
        \texttt{\$1000 - \$1FFF} & ram      & ram   & ram      & ram       & CHAR  \\
        \texttt{\$0000 - \$0FFF} & ram      & ram   & ram      & ram       & ram   \\
        \bottomrule
    \end{tabularx}
    \caption{C64 memory configurations with LHGX = 0100}
    \label{tab:mem0100}
\end{table}

\begin{table}[!h]
    \centering
    \begin{tabularx}{0.9\textwidth}{>{\centering}X|c|c|c|c|c}
        \toprule
        \multicolumn{6}{c}{\#LORAM 1, \#HIRAM 1, \#GAME 1, \#EXROM 0} \\
        \midrule
            & \multicolumn{2}{c|}{\#CHAREN 1} & \multicolumn{2}{c|}{\#CHAREN 0} & \\
        Address         & CPU R & CPU W & CPU R & CPU W & VIC R \\
        \midrule
        \texttt{\$F000 - \$FFFF} & KERNAL   & ram   & KERNAL   & ram       & ram   \\
        \texttt{\$E000 - \$EFFF} & KERNAL   & ram   & KERNAL   & ram       & ram   \\
        \texttt{\$D000 - \$DFFF} & I/O\footnotemark[1] & I/O & CHAR     & ram       & ram   \\
        \texttt{\$C000 - \$CFFF} & ram      & ram   & ram      & ram       & ram   \\
        \texttt{\$B000 - \$BFFF} & BASIC    & ram   & BASIC    & ram       & ram   \\
        \texttt{\$A000 - \$AFFF} & BASIC    & ram   & BASIC    & ram       & ram   \\
        \texttt{\$9000 - \$9FFF} & ROML     & ram   & ROML     & ram       & CHAR  \\
        \texttt{\$8000 - \$8FFF} & ROML     & ram   & ROML     & ram       & ram   \\
        \texttt{\$7000 - \$7FFF} & ram      & ram   & ram      & ram       & ram   \\
        \texttt{\$6000 - \$6FFF} & ram      & ram   & ram      & ram       & ram   \\
        \texttt{\$5000 - \$5FFF} & ram      & ram   & ram      & ram       & ram   \\
        \texttt{\$4000 - \$4FFF} & ram      & ram   & ram      & ram       & ram   \\
        \texttt{\$3000 - \$3FFF} & ram      & ram   & ram      & ram       & ram   \\
        \texttt{\$2000 - \$2FFF} & ram      & ram   & ram      & ram       & ram   \\
        \texttt{\$1000 - \$1FFF} & ram      & ram   & ram      & ram       & CHAR  \\
        \texttt{\$0000 - \$0FFF} & ram      & ram   & ram      & ram       & ram   \\
        \bottomrule
    \end{tabularx}
    \caption{C64 memory configurations with LHGX = 1110}
    \label{tab:mem1110}
\end{table}

\begin{table}[!h]
    \centering
    \begin{tabularx}{0.65\textwidth}{>{\centering}X|c|c|c}
        \toprule
        \multicolumn{4}{c}{\#LORAM x, \#HIRAM x, \#GAME 0, \#EXROM 1} \\
        \midrule
            & \multicolumn{2}{c|}{\#CHAREN x} & \\
        Address         & CPU R & CPU W & VIC R \\
        \midrule
        \texttt{\$F000 - \$FFFF} & ROMH     & ROMH      & ROMH  \\
        \texttt{\$E000 - \$EFFF} & ROMH     & ROMH      & ram   \\
        \texttt{\$D000 - \$DFFF} & I/O\footnotemark[1] & I/O & ram   \\
        \texttt{\$C000 - \$CFFF} & -        & -         & -     \\
        \texttt{\$B000 - \$BFFF} & -        & -         & ROMH  \\
        \texttt{\$A000 - \$AFFF} & -        & -         & -     \\
        \texttt{\$9000 - \$9FFF} & ROML     & ROML      & ram   \\
        \texttt{\$8000 - \$8FFF} & ROML     & ROML      & ram   \\
        \texttt{\$7000 - \$7FFF} & -        & -         & ROMH  \\
        \texttt{\$6000 - \$6FFF} & -        & -         & -     \\
        \texttt{\$5000 - \$5FFF} & -        & -         & -     \\
        \texttt{\$4000 - \$4FFF} & -        & -         & -     \\
        \texttt{\$3000 - \$3FFF} & -        & -         & ROMH  \\
        \texttt{\$2000 - \$2FFF} & -        & -         & -     \\
        \texttt{\$1000 - \$1FFF} & -        & -         & -     \\
        \texttt{\$0000 - \$0FFF} & ram      & ram       & ram   \\
        \bottomrule
    \end{tabularx}
    \caption{C64 memory configurations with LHGX = xx01}
    \label{tab:mem001x}
\end{table}

%%%%%%%%%%%%%%%%%%%%%%%%%%%%%%%%%%%%%%%%%%%%%%%%%%%%%%%%%%%%%%%%%%%%%%%%
% http://tim.thorpeallen.net/Courses/Reference/Citations.html

\begin{thebibliography}{999}

\bibitem [PRG83]{PRG83} Commodore Computers, 1983, \textit{Commodore 64
Programmer's Reference Guide}, Financial Times Prentice Hall.

\bibitem [Lev86]{Lev86} William Levak and Ann Arbor, 1986, Commodore 64 Memory
Configurations, \textit{The Transactor}, Volume 6, Issue 05, p. 51-55.

\bibitem [PLA95]{PLA95}
http://www.zimmers.net/anonftp/pub/cbm/firmware/computers/c64/pla.txt

\bibitem [Herd12]{Herd12} Mail by Bil Herd to cbm-hackers mailing list,
http://www.softwolves.com/arkiv/cbm-hackers/17/17534.html

\bibitem [AC12]{AC12} http://www.arcadecomponents.com/downloads.html

\end{thebibliography}

\end{document}

